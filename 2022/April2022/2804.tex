\subsection{28. April}
Heute habe ich mich zunächst mit dem Abschluss der gestern diskutierten Pull-Request, sowie neuen Kommentaren unter einer weiteren Pull-Request beschäftigt. Dank eines frischen Kopfes konnte ich hier schnell Fortschritte machen. Allerdings habe ich beim Bearbeiten dieser Requests einen nicht von mir erzeugten Fehler festgestellt, und konnte diesen auch auf dem Qualitätssicherungssystem reproduzieren. Ich habe daher eine grobe Analyse durchgeführt, um die grobe Ursache des Fehlers festzustellen, und ein Ticket auf dem Jira-Board dafür erstellt. Anschließend haben mein Vorgesetzter und ich zusammen untersucht, ob der Fehler auf der Seite der .Net API auftritt, was nicht der Fall war. Deshalb habe ich mich in der PHP API auf die Suche gemacht, und konnte dort in einer internen Methode die Ursache des Fehlers finden. Ich habe eine schnelle Lösung gefunden, die jedoch zunächst pausiert wurde, da das Auftreten des Fehlers auf eine grundlegendere Problematik hinweist. Ich habe dann meine Tätigkeit hieran pausiert, um mich mit dem Erstellen einer scss Datei für das Überschreiben der Styles einer typo3-core-extension auseinanderzusetzen. Hierbei konnte ich mich wieder in unser static-Projekt und die Arbeit mit scss und grunt einfinden, was eine willkommene Auffrischung war.
Am Nachmittag haben wir eine Sprint-Retrospektive durchgeführt. Da ich in der Berufsschule gerade erst einiges über Projektmanagement gelernt habe, und sich die zentralen Themen stark im Bereich des Team-Managements und Scrum Prozesses bewegt haben, habe ich versucht mich intensiv einzubringen. Ich wurde in eine Kleingruppe aufgenommen, um die möglichen Aufgaben eines Scrum-Masters zusammenzutragen und hoffe, hier viele lehrreiche Einblicke gewinnen zu können.