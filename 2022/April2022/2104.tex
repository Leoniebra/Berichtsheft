\subsection{21. April}
Heute bin ich von der Berufsschule in den Betrieb zurückgekehrt. Da wir heute einen neuen Sprint gestarted haben, war es zum Einstieg etwas schwierig, eine geeignete Tätigkeit für mich zu finden. Ich habe dann eine Anfrage zugewiesen bekommen, in der einer unserer Systemintegratoren um das Umstellen des HTTP IIS Bindings meiner Türkontrollenapp auf eine sicherere Methode gebeten hatte. Ich habe mich daraufhin in das Konzept eingelesen, und im IIS Konfigurationstool eine entsprechende Einstellung gefunden.Im Gespräch mit dem Mitarbeiter habe ich dann gelernt, dass das IIS Binding entweder code-seitig oder konfigurativ eingestellt werden kann. Im Falle meiner .Net API wäre das dann im Rahmen der web.config datei geschehen. Da ich dieses Binding aber bei der Entwicklung nicht explizit eingebaut hatte, konnten wir konfigurativ die Anwendung auf HTTPS umstellen.

Weiter habe ich nach dem Sprintplanungsmeeting gemeinsam mit meinem Supervisor eine eigene Entwicklungsumgebung auf einem der dev Server eingerichtet. Hierfür haben wir in mein durch einen ssh-tunnel erreichtes schon existierendes Git Laufwerk das repository geklont, Typo 3 dort installiert und die Datenbank kopiert. Ich habe anschließend begonnen, die nötigen Arbeitsschritte in einem Dokumentationsartikel zusammenzufassen um die Einarbeitung zukünftiger Mitarbeiter:innen zu erleichtern.