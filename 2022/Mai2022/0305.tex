\subsection{03. Mai}
Heute habe ich mich intensiver mit dem Zusammenbau der Request-Objekte auseinandergesetzt. Ich habe die Verwendung von Setter Funktionen mithilfe von Reflection automatisiert, so dass ich konfigurativ nur noch eine Liste von Eigenschaften, sowie deren Zuordnung zu Formularfeldern anpassen muss. Meine Logik findet dann über Reflection die Methodennamen der zugehörigen setter-funktionen und besetzt die Eigenschaften mit den aus dem Formular ausgelesenen Feldern. 
Beim weiteren Debuggen durch die .net API ist mir aufgefallen, dass mein Request-Objekt unvollständig war, wodurch einige sehr lange Methodenlaufzeiten ausgelöst wurden, die im live-system sicher für einen timeout im frontend geführt hätten. Dies hatte daran gelesen, dass die zu einem newsletter opt-in gehörende Email-adresse im request objekt mehreren Feldern zugewiesen werden muss. Ich konnte dieses Problem mit der Hilfe eines Mitarbeiters finden und dann selbstständig beheben. Schließlich habe ich es geschafft, aus dem Frontend eine Abfrage abzusetzen, und den gesetzten Opt in dann auf der Datenbank des CRM-Systems zu finden. Das fand ich sehr cool, weil ich hiermit durch den vollen Stack zweier Entwicklungsteams hindurch eine Anfrage 'komplett durchgereicht' habe.
Schließlich habe ich mit Spamabwehrmechanismen auseinandergesetzt. Ich habe ein Honeypot-Feld in das Formular integriert, bei Ausfüllung dessen der Serverseitige Prozess sofort abgebrochen wird. Weiter habe ich mich mit der Weitergabe zeitlich begrenzter Tokens zur Sicherstellung dass inputs nur über den tatsächlichen Aufruf des Formulars eingehen auseinandergesetzt.