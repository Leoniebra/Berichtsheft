\subsection{23. Mai}
Heute habe ich mich intensiv mit dem erstellen von Unit Tests für PHP und TYPO3 beschäftigt. Ich bin zunächst in die schwierigkeit gelaufen, dass TYPO3 einiges an Dependency Injection im Hintergrund verwendet, welche nicht über explizite Angabe von Konstruktorargumenten aufgelöst wird. Dies hat selbst bei korrekter Verwendung von Mocks das instanziieren von Klassen und damit das testen nicht statischer Methoden sehr erschwert. Allerdings konnte ich für statische Methoden bereits einige unit tests entwickeln, und konnte mithilfe von reflection hierbei auch private Methoden testen. Ich habe tests für verschiedene Fälle in welchen ich mithilfe eines regulären Ausdrucks eine URL durchsuche geschrieben, welche alle erfolgreich waren.
Ich habe dann den Ansatz reflection zu verwenden weiter getrieben, um Klassen ohne expliziten Aufruf des Konstruktors zu instanziieren. Hierdurch konnte ich tatsächlich einen Test für die Hauptmethode meines entwickelten Commands schreiben, habe jedoch noch keinen guten Weg gefunden, die Argumente mit denen die Methode, welche die geänderten Objekte in die Datenbank schreibt aufgerufen wird, zu speichern und mit meiner Erwartung zu vergleichen.
Ich habe einen Confluence Artikel erstellt, in dem ich meinen bisherigen Fortschritt zusammengefasst habe, und in einem kommenden Teammeeting eine Aufgabe erstellt, um meine Teammitglieder und mich daran zu erinnern hierfür ein gemeinsames Brainstorming vorzunehmen.