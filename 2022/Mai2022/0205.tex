\subsection{02. Mai}
Heute habe ich mit dem Schreiben der konkreten Logik in meiner finisher-Klasse begonnen. Hierzu habe ich konfigurativ einige Felder im Formular als datentragende Felder identifiziert, und aus diesen ein Request-Objekt für einen Soap-Call an die .Net API zusammengebaut. Hierzu habe ich einen Testdurchstich verwendet, der es mir ermöglicht anstelle der auf dem Entwicklungsserver laufenden API eine lokal auf meinem Rechner laufende API mit der Website zu verbinden. Hierzu kann ich die über den Soap Call an die API weitergereichten Objekte im .net Projekt nachverfolgen, um sicherzugehen dass alle nötigen Informationen bereitgestellt wurden. Hierzu existiert leider keine einheitliche Dokumentation, und einige Felder werden redundant besetzt.

Beim Einrichten des Testdurchstiches ist mir aufgefallen, dass ich die .Net Solution seit dem Upgrade auf .Net Core, das vor einigen Wochen durchgeführt wurde nicht mehr kompiliert hatte, und ich musste mir von einem Mitarbeiter dabei helfen lassen die Konfiguration meiner IDE entsprechend anzupassen. Ich konnte zum Ende des Tages jedoch erfolgreich einen Soap Call absetzen, und die von mir abgesetzte Anfrage auf der lokalen API eingehen sehen.