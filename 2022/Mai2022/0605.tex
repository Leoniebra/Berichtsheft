\subsection{06. Mai}
Heute habe ich mich weiter mit dem Problem von gestern befasst. Ich habe das Problem einem Mitarbeiter gezeigt, der mir empfohlen hat, zu versuchen die UI-Elemente per Javascript zu laden, da dies der Weg ist auf dem die extension sie standardmäßig läd. Leider habe ich hiermit eine Menge neuer Fehler generiert. Zunächst hat sich herausgestellt, dass eine falsche Schreibweise in einer Pfadangabe in der Yaml-Konfiguration dazu geführt hatte, dass meine eingebundene Javascript-Datei von der Content Security Poliy (CSP) von TYPO3 geblockt wurde. Diesen Fehler konnte ich erst mit Hilfe eines Mitarbeiters beheben, weil ich intuitiv überhaupt nicht auf den Zusammenhang zwischen der Konfiguration meiner Extension und der CSP der Seite gekommen war. Scheinbar hatte meine Schreibweise (wir verwenden eine Kurzschreibweise für Pfade) zufälligerweise die Syntax von CSP-Tags kopiert und dadurch einen interpretierungsfehler verursacht. Im Anschluss hieran bin ich auf ein weiteres Problem gestoßen: Zwar konnte ich die Datei jetzt auf meine Seite laden, aber es wurde beim Ausführen des inneren JavaScriptes in jedem Fall ein Scriptfehler ausgelöst - selbst wenn die Datei kein Script enthielt. Löste ich die Referenz zu der Datei auf, behob sich jedoch auch der Fehler. Ich konnte noch keine funktionierende Lösung für dieses Problem finden, und musste feststellen dass meine gestrige Krankheit doch noch nicht völlig genesen war, wodurch sich mein Fortschritt ebenfalls verzögern musste.