\subsection{05. September}
Heute habe ich das Problem der vermeintlich durch meine Änderungen nicht mehr funktionierenden Unterseite gelöst. Hierbei stellte sich hieraus, dass nicht ich die Seite kaputt gemacht habe, sondern mit dem Update auf PHP 8.1 die Signatur einer Methode aus einer PHP-Internen Utility-Funktion (der number-formatter) strenger geworden war, und inzwischen für eine übergebene Zahl fest entweder string oder float als Datentyp fordert. Hierdurch wurde eine Exception geworfen, wenn wir aus einem der ViewHelper einen string an diese Funktion übergeben haben. Das Problem konnte durch ein explizites Type casting behoben werden. Ich habe im Anschluss hieran weiter durch PHPStan angemerkte Fehler bearbeitet. Außerdem habe ich mein Berichtsheft von overleaf.com zu einem lokalen Ordner migriert, um es mir zu erleichtern eine Versionskontrolle mittels GitHub einzuführen. Ich habe dazu VIM und MikTeX installiert, und einige Aliases in PowerShell erstellt. Am Nachmittag hat ein Mitarbeiter mich beim Update des QS-Servers über seine Schulter blicken lassen. Ich habe hierbei einige weitere Tricks für den Umgang mit Linux und Docker erfahren, und coole Einblicke in Nginx sowie die administrative Arbeit im Zusammenhang mit der Aufrechterhaltung einer modernen Infrastruktur erhalten.
Es hat sich außerdem beim Umzug des Berichtsheftes herausgestellt, dass meine git Version die falschen Zertifikate verwendete, und ich hierdurch keine HTTPS-Verbindung zu GitHub aufbauen konnte. Diese Probleme konnte ich mithilfe meines Kollegens überwinden, der mich dann auch gleich noch dabei unterstützt hat, die relevanten Zertifikate auch in WSL zu hinterlegen.
