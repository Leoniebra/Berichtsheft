\subsection{16. September}
Heute habe ich ebenfalls die CodeTalks Konferenz besucht. Gleich als erstes am Morgen habe ich einen spannenden Talk zu Mob-Programmierung gehört. Dieses Konzept erweitert die Idee des Pair-Programmings auf Programmierungssessions an denen ein ganzes Team beteiligt ist. Die Vorteile dieser Methode, sowohl für die Produktivität als auch für den Teamzusammenhalt haben mich überzeugt, dieses Konzept auch in meiner Firma vorzuschlagen. Ich hatte bereits vor der Konferenz mit einem Mitarbeiter über die Perspektive diesen Talk zu besuchen gesprochen, und er war ebenfalls der Möglichkeit gegenüber aufgeschlossen. An diesen Talk anknüpfend habe ich einen weiteren Talk zu Trunk based development gehört, einer Arbeitsweise die TDD und mob- bzw pair-programming sowie die inkrementelle Auslieferung von features vereint, um source code reviews obsolet zu machen und einen schnelleren, und sichereren Entwicklungsprozess zu gewährleisten. Ich war sehr beeindruckt von diesem Vortrag, und während ich eine direkte Umsetzbarkeit in meinem Betrieb noch nicht sehen kann, finde ich doch dass einzelne Aspekte dieser Methode (wie TDD und pair programming, sowie die grundlegende Philosophie beim Schneiden von User-Stories) erwähnens- und umsetzenswert auch in meinem Team sind. Ich nehme mir als festen Vorsatz mit, meine Mitarbeiter:innen aktiv zu pair programming sessions einzuladen, und auch für mob-programming zu werben. Während die folgenden Talks leider nicht ganz so einsichtsreich waren (ein interessanter Talk zur Philosophie des agilen Arbeitens verdient Erwähnung, jedoch war die Zielgruppe management auf hohem Level), konnte ich den Tag mit einem sehr spannenden Talk abschließen: Ein Entwickler hat git pre-commit Hooks, und ein Framework zur Versionierung und Management dieser vorgestellt. Während das Framework für uns vermutlich nicht nutzbar ist, bieten pre-commit hooks eine tolle Möglichkeit, meine Arbeit mit phpstan der letzen Wochen in den Workflow des teams einzubinden. Ich habe außerdem vor, den Autoformatter von PHPStorm ebenfalls in einem pre-commit hook über meine Änderungen laufen zu lassen, um so peinliche Kommentare und Fehler beim Code-Review zu reduzieren.