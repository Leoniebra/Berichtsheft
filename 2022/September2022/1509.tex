\subsection{15. September}
Heute habe ich statt der Berufsschule eine IT-Konferenz - die CodeTalks in Hamburg - besucht. Ich habe talks zu verschiedenen Themen gehört, von welchen ich manche hoffe in der Arbeit im Betrieb einbringen zu können. Besonders ist mir hierbei ein Talk zu test-driven development in Erinnerung geblieben, der anhand von einem Live-Coding example die Philosophie und den Mehrwert von Test-driven Development illustriert hat. Ich hoffen, mit diesen neuen Erkenntnissen eine bessere Position als Botschafterin für TDD in meinem Betrieb einnehmen zu können, und meine Mitarbeiter:innen vom Mehrwert dieser Methode überzeugen zu können. Da ich selbst schon an der Erstellung und Einführung von Unit-Tests mitgearbeitet habe war das sehr aufschlussreich für mich. Ich habe außerdem spannende Talks zum Monitpring, Kubernetes Probes, Teamarbeit zwischen UX-Designer:innen und Entwickler:innen, asynchroner Programmierung in PHP und der Verwendung von Rabbit MQ als Task-scheduling System und mathematischer Modellierung gehört. Während diese Talks keine so unmittelbare Anwendung in meiner aktuellen Arbeit finden können, so haben sie doch meinen Horizont und mein übergreifendes Verständnis für die momentane IT-Landschaft und moderne Technologien erweitert.