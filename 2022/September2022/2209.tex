\subsection{22. September}
Heute haben wir mit der Bearbeitung des Lernfeldes begonnen. Da wir nur wenig Zeit übrig haben, und die Motivation in meiner Gruppe gering war, haben ich eine einfach zu implementierende Architektur zur Lösung des Lernfeldes vorgeschlagen: Das Lernfeld beinhaltet das Überwachen der Hardwareressourcen eines Servers, und das Erstellen von Alerts, sollten diese Ressourcen kritische Grenzwerte überschreiten. Mein Vorschlag war, hierzu zwei unabhängige Services zu erstellen, die über eine Rest-API kommunizieren. Der eine Service überwacht den Server, konsolidiert die Messdaten und erstellt bei Bedarf einen Alert, der per HTTP-GET abgefragt werden kann. Der andere Service holt in regelmäßigen Abständen diese Alerts ab verarbeitet diese (gemäß hinterlegbarer Konfigurationen), um sie anschließend an den gewünschten Benachrichtigungshook weiterzugeben.
Ich habe den großteil des Tages damit verbracht, in Partnerarbeit mit einem meiner Mitschüler der noch keine Entwicklungserfahrung hat eine Entwicklungsumgebung einzurichten, git zu installieren, ein leeres git repository zu initialisieren und auf gitHub zu teilen, eine Hello-World Seite in php zu erstellen und diese mithilfe des php8.1-apache images in docker zu deployen. Wir versuchen auf diese Art und Weise eine CI-Pipeline mit einfachsten Mitteln (und ohen Automation) zu realisieren, um unseren anderen beiden Teamkollegen möglichst schnell ein Interface zur Verfügung zu stellen, gegen das sie testen können.