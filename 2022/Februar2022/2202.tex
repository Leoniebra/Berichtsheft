\subsection{22. Februar}
Heute habe ich erfolgreich den Testfont in das Projekt eingebunden. Ich hatte hierzu den Rat eines Mitarbeiters befolgt, und zunächst eine stark vereinfachte Lösung implementiert, die ich dann sukzessive dem gegebenen Komplexitätsniveau angepasst habe. Anschließend habe ich den Testfont durch den durch die Akzeptanzkriterien der User-Story gewünschten Font ersetzt. Im Anschluss hieran habe ich mich mit der Einbindung von Fonts mit mehrfarbigen Glyphen auseinandergesetzt. Hier habe ich feststellen müssen, dass dies deutlich komplexer ist als die Verwendung von einfarbigen Glyphen, und für den gegebenen Anwendungsfall nicht ohne die Verwendung von Javascript realisierbar ist. Ich habe mich deswegen entschieden, das zweite Akzeptanzkriterium nicht durch Nutzung eines Fonts, sondern durch die direkte Einbindung von base64-enkodierten Bilddateien im before-tag des Listenelements zu lösen. Hier gibt es noch einiges an Konfigurationsarbeit zu leisten, befor das gewünschte Styling erzielt ist.