\subsection{14. Februar}
Heute habe ich mit einem Mitarbeiter über die von mir gestellte Pull-Request zum Einfügen eines Hinweistextes beim Ändern von Email-adressen gesprochen. Hierbei hat sich herausgestellt, dass der von mir geschriebene JavaScript Code einen logischen Fehler hatte, da ich ein Element das nicht notwendigerweise existiert referenziert habe. Weiter haben wir uns darüber Gedanken gemacht, ob auch eine JavaScript-freie Lösung möglich wäre. Wir sind dabei verblieben, dass ich verschiedene Lösungsansätze implementieren soll, und mich mit der Fachabteilung bezüglich der Auswahl einer konkreten Implementierung in Verbindung setzen werden. Die Lösungen die ich im Endeffekt implementiert habe sind die Folgenden: Erstens habe ich anstelle eines on-click events ein input event implementiert, so dass die textbox nicht nur bei Navigation per Maus angezeigt werden kann. Als Alternative habe ich ein popover implementiert, in welchem der Hinweistext angezeigt wird. Final habe ich eine noscript Lösung implementiert , um auch Nutzer:innen die die Verwendung von JavaScript nicht erlauben einen Hinweis anzeigen zu können.
Im Anschluss hieran habe ich mich mit der Untersuchung eines SoapCall Fehlers beschäftigt, und bin tiefer in die Controllerarchitektur des Projektes eingestiegen, konnte allerdings noch keine signifikanten Fortschritte machen.