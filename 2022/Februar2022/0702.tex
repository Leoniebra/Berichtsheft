\subsection{07. Februar}
Heute habe ich mit der Bearbeitung von Aufgaben aus dem Scrum-Board des aktuellen Sprints begonnen. Zunächst habe ich eine Pull-Request für das Hinzufügen eines fehlenden Titels im Typo3-Backend gestellt. Dieser Titel hat wegen einer fehlenden Lokalisierungsdatei gefehlt, welche zu ersetzen mir auch die möglichkeit gegeben hat, einen Teil des Arbeitscodes auf einen neueren Standard zu bringen. Im Anschluss daran habe ich ein SQL-Skript geschrieben, um auf der Datenschutzseite unserer Firma Erwähnungen der Firma Facebook durch Meta zu ersetzen. Hierbei gab es etwas Abstimmungsbedarf mit der Fachabteilung, weil die von ihnen genannten Ersetzungsregeln inkonsistent waren. Schließlich habe ich begonnen, mich mit dem Hinzufügen eines Hinweises zum Widerrufsrecht auf der Opt-In Seite zu beschäftigen. Hierbei bin ich in ein Problem gelaufen, weil der Versuch, die Opt-In Seite auf dem Dev system aufzurufen zu einem Typo3-Error geführt hatte. Ich konnte mich auch nicht mit einem Testdurchstich durch eine lokale API verbinden. 