\subsection{29. August}
Heute habe ich bei der Bearbeitung der von PHPStan festgestellten Fehler leider kaum Fortschritte machen können. Ich habe zwar einige kleinere (dennoch spannende) refactorings in den Code eingebaut, mein Fortschritt wurde jedoch stark durch die Anzeige von Fehlern gehemmt die ich weder nachvollziehen, noch testen konnte. Da der aktuelle Arbeitsbranch bereits PHP 8.1 benutzt, das Entwicklungssystem aber noch PHP 7.4 verwendet kann ich momentan nicht auf dem gewohnten Entwicklungsserver testen. Zwar steht ein weiterer Server mit PHP 8.1 bereit, hier musste ich jedoch zunächst einiges an Konfigurationsarbeit vornehmen, um eine SSH-Verbindung zu unserem git-System aufbauen zu können. Im Rahmen dieser Konfiguration habe ich über Dateizugangskontrolllisten (\textit{facl}) gelernt. Diese fügen Dateien weitere, über die grundlegenden Berechtigungen für User, Gruppe und Andere hinausgehende Berechtigungen hinzu. Mir ist das über aufgestoßen, da ich versucht hatte einen private key auf dem Server zu hinterlegen, und SSH sich weigerte diesen zu benutzen bevor ich einer bestimmten Gruppe die Leserechte entzogen hatte. Ich habe es schließlich geschafft, die SSH verbindung einzurichten, und konnte auf diesem Weg noch einiges über Linux lernen.

Weiter habe ich mich mit dem Übersetzen von zwei Powershell Skripten meines Mitarbeiters in ZSH beschäftigt. Zwar sind powershell und zsh sehr ähnlich, trotzdem haben die unterschiedlichen Dialekte mir einige Schwierigkeiten bereitet. Ich konnte allerdings durch die Arbeit an diesen Skripten und die Automation von Git-Prozessen auf diesem Weg einiges dazulernen, und mein Umgang mit git aus der Kommmandozeile ist hierdurch um einiges vertrauter geworden.