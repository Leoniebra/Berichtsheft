\subsection{15. August}
Heute habe ich ein Diagramm für die Systemlandschaft der für das Monitoring herangezogenen Systeme erstellt. Hierdurch konnte ich selbst den Weg, den die Metrik- und Logdaten in den Monitoringsystemen nehmen noch einmal selbst besser verstehen und nachverfolgen, und beim Gespärch mit meinem Mitarbeiter der der Hauptakteur bei der Implementierung und Einrichtung dieser Systeme gewesen ist sind uns beiden weitere Möglichkeiten der Nutzung und Vernetzung dieser Systeme in den Sinn gekommen. Was uns besonders gefallen und ins Auge gefallen ist, war dass sowohl Prometheus, als auch Loki und Elasticsearch alle Rest-APIs zur Verfügung stellen, die unter Anderem auch von den beobachteten Anwendungen konsumiert werden könn(t)en. Dies öffnet Möglichkeiten dafür, auch innerhalb der Applikationslogik den Kontext des aktuellen Status der Systeme zu berücksichtigen.