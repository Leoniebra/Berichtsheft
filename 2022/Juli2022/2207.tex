\subsection{22. Juli}
Heute habe ich mich genauer mit der Erstellung von Queries in der Query-DSL (Designated Search Language) von Elastic auseinandergesetzt. Ich konnte mithilfe von einigen Youtube-Tutorials einige praktische Queries an den von meinem Mitarbeiter importierten Log-Datensatz stellen. Weiter habe ich mich mich mit der Erstellung und Manipulation von Dashboards beschäftigt. Ich konnte zwar ein zeitaktuelles Log-Dashboard anlegen, jedoch gab es keine Konfigurationsmöglichkeiten für die angezeigten Spalten. ich habe deshalb begonnen, mich in die Grafik-Grammatik Vega einzuarbeiten, die zur freien GEstaltung von Dashboards in Kibana genutzt wird. Leider hat die Oberfläche in Kibana dazu keinen vernünftigen Debug-Modus, so dass ich viel im Trüben fischen musste. Im Rahmen dieser Grafik-Grammatik definiere ich eine HTTP-Anfrage an die Elastic search-API, nehme Transformationen an dem erhaltenen Response-Objekt vor und definiere anschließend dessen grafische Darstellung. Diese eigentlich sehr unkomplizierte Aufgabe ist durch die fast vollständige Verschleierung der manipulierten Objekte zu einer nachmittagsfüllenden Aufgabe geworden. 