\subsection{18. Juli}
Heute habe ich meine PHP Anwendung zur exemplarischen Generierung von Metriken finalisiert, und mit meinem Mitarbeiter einige grundlegende Aspekte der Architektur, sowie einige wenige oberflächliche Fehler im Code besprochen. Ich habe begonnen, einen Confluence-Artikel der Entwickler:innen als Einstigeshilfe beim Export von Metriken aus ihren Anwendungen dienen soll zu verfassen. In diesem Artikel verwende ich die von mir entwickelten Anwendugnen als Beispiele. Was mir beim Schreiben dieses Berichtsheftseintrages auffällt ist, dass eine Verknüpfung mit Grafana in diesem Artikel und das exemplarische Zeigen von zu den Metriken gehörenden Dashboards sicherlich ein wertvolles Addendum zu dem Artikel wäre.
Ich habe außerdem an einem Meeting teilgenommen, in welchem von einer Versicherung gestellte Anforderungen zu Datensicherungs- und Backup-Standards besprochen wurden, die sich insbesondere auf die Dokumentation von Wiederherstellungsprotokollen im Fall des Versagens einer oder mehrerer hauseigener Systeme bezogen. Hierdurch konnte ich den Wert von ausführlicher und lesbarer Dokumentation auch für ein effektives Risikomanagement im IT-Bereich und die Entkopplung von Aufgaben und Personen an einem direkten Beispiel sehen. Im Anschluss an dieses Meeting hat ein Mitarbeiter einem weiteren Mitarbeiter und mir ein kurzes Einführungsseminar in die Bedienung und Struktur von Linux, sowie die Verwendung einiger hauseigener cmdlets gegeben.
Zum Ende des Tages habe ich meine C\#-Anwendung zur Generierung von Daten erweitert, da momentan auf dem Entwicklungsserver viel ungenutze Rechenkapazität zur Verfügung steht. Anstelle den random-walk eines Teilchens zu simulieren, simuliere ich nun $10^4$ Teilchen, wodurch sich innerhalb kurzer Zeit ein interessanter Datensatz generieren lässt.