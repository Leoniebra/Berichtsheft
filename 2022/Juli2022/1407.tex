\subsection{14. Juli}
Heute habe ich die Anwendung zur Bereitstellung von Prometheus-Metriken die ich in C\# geschrieben hatte als serverseitige PHP-Api nachgebaut. Da die gängigen PHP-Bibliotheken zum Erstellen solcher Anwendungen ein großes Framework zur Ausführung von PHP Serverprozessen beinhalten, und unsere Abteilung nur einen bei Aufruf ausgeführten PHP code nutzt, habe ich mich entschieden ein grundlegendes framework für diesen Anwendungsfall from scratch zu bauen. Hierzu habe ich mich stark an der C-Bibliothek orientiert, um die dort bereitgestellte Funktionalität zu implementieren. Mein Mitarbeiter hat die domain meiner PHP API ebenfalls im Grafana registriert, so dass ich die Kompatibilität der von mir bereitgestellten Daten in Echtzeit prüfen konnte. Das entwerfen der Klassenstruktur für die Metriken hat sich als umfangreicher herausgestellt, als ich ursprünglich erwartet hatte, war jedoch eine tolle Übung für den effizienten Aufbau von Klassenstrukturen.