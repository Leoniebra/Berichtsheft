\subsection{13. Juli}
Heute habe ich meine Arbeit mit ActiveDirectory und Powershell vorerst abgeschlossen, und mit dem nächsten Größeren Interessenfeld meiner Hospitation gewidmet: Dem Monitoring-System. Ich habe zunächst eine .net Core - Konsolenanwendung gebaut, die Metriken für das Zeitserienbasierte Datenbanksystem Prometheus bereitstellt. Hierzu habe ich mich der .net-Library von Prometheus bedient, welche über den NuGet Package-Manager verfügbar ist und dies zu einer leichten Übung gemacht hat. Ich habe Beispiele für die vier von Prometheus unterstützten Metriken erstellt: Counter, Gauges, Summaries und Histograms. Diese habe ich dann mit Daten befüllt und über eine HTTP schnittstelle bereitgestellt. Mein Mitarbeiter hat meine lokale Entwicklungsumgebung dann im Firmeninternen Grafana registriert, wo ich die Möglichkeit hatte die von mir erzeugten Daten grafisch aufzubereiten. Ich habe als Datenmodell eine eindimensionale Monte-Carlo Simulation für Diffusion (also einen random walk) verwendet. Im Anschluss an das Verknüpfen von Grafana mit meiner API hat mein Mitarbeiter mir die Serverstruktur von Prometheus, sowie das Alerting via Prometheus-Alert erklärt, und wir haben die im Haus genutzte Implementierung, sowie die Konfiguration der Services mittels Yaml besprochen. Schließlich hat mein Mitarbeiter die von mir erstellte Anwendung mithilfe von Docker auf einen der Entwicklungsserver deployed und mir die Nutzung des docker-Containers erklärt.