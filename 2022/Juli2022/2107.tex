\subsection{21. Juli}
Ich habe mich heute verstärkt mit dem Vergleich zwischen ElasticSearch und Grafana Loki befasst. Mein Mitarbeiter hat auf dem Enwicklungsserver eine Instanz von Elastic und Kibana in einem Docker Container deployed, und mir dort ermöglicht Kibana und ElasticSearch mit Testdaten auszuprobieren. Wir haben im Anschluss daran über Probleme gesprochen, die mir beim Umgang mit dem Systemen aufgefallen sind. Es stellte sich heraus, dass meine größten bedenken (die fehlende Möglichkeit zur Prä-prozessierung von Logs, sowie die hinter einem paywall gehaltene Andbindung der Alerting-Funktionalität an MS teams und Mail) durch third Party Lösungen behoben werden können. Positiv fiel mir auf, dass die grafische Oberfläche von Kibana und die vereinfachte top-level query sprache einen sehr leichten Einstieg in den Umgang und die Erkundung gesammelter Daten liefern. Ich habe einen großen Teil des Tages damit verbracht, mir Tutorials und Einführungen in die beiden Systeme anzusehen und konnte auch einiges des gelernten unmittelbar anwenden.
In einem langen Video zur Nutzung und Vorstellung von Grafana Loki konnte ich bereits einige fundamentale Unterschiede zwischen den Systemen feststellen. Für den kommenden Tag wird des zur Aufgabe, auch Loki ausführlicher zu testen, sowie die beiden Testsysteme mit echten Daten aus unseren Systemen zu befüllen um aussagekräftigere Aussagen aus den Datensätzen zu extrahieren.