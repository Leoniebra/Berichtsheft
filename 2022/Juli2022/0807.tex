\subsection{08. Juli}
Heute habe ich mich intensiv mit dem Modifizieren und Erstellen von Skripten in Powershell auseinandergesetzt. Hierzu habe ich zunächst ein Skript, welches wir zum Erstellen eines Git-Branches auf Basis eines Jira-Tickets verwenden bearbeitet, um einen müßigen Konsolenaufruf zum Starten des Skriptes zu vereinfachen. Ich habe hierzu ein interaktives Auswahlelement mithilfe der Out-GridView Funktion erzeugt, und eine dynamische Navigation in den Zielordner implementiert.
Im Anschluss hieran habe ich ein Powershell Skript, das eine JSON-Datei mit Email-Adressen ausliest, und an alles in der Datei vermerkten Empfänger:innen eine Erinnerungsemail zum Bestellen von Essen in unserer Kafeteria in der Folgewoche verschickt erstellt. Hierdurch konnte ich den Umgang mit Kommunikationsobjekten und der Outlook-Schnittstelle, sowie das Erstellen von SchedulableTasks via Powershell lernen. Ich habe im Anschluss mit einem Mitarbeiter die Sktipte die bei Start des Computers sowie Login eines Users ausgeführt werden besprochen, und begonnen eine Active-Directory Abfrage in Powershell umzusetzen.