\subsection{19. Juli}
Heute habe ich mich mit der Erstellung von Dashboards in Grafana für die Auswertung der von meiner Beispielanwendung erzeugten Daten befasst, und meine Arbeit an Dokumentationsartikeln zu Erstellund und Export von Metriken für Prometheus, sowie der von mir erstellten Klassenstruktur für die PHP Anwendung befasst. Ich habe außerdem meine bisherigen Projekte in Git Repositories mit meinem Team geteilt.
Im Anschluss hieran haben mein Kollege und ich über eine folgende Aufgabe für mich gesprochen. Wir wollen nun die Arbeit am Monitoring-System fortsetzen und uns von der veröffentlichung und speicherung von Metriken hin zur Speicherung und Verarbeitung von Events und Logs wenden. Hier besteht noch keine etablierte Infrastruktur bei uns, und meine erste Aufgabe war es, verschiedene Softwarelösungen zu vergleichen. Die beiden zentralen Alternativen sind hierbei ElasticSearch und OpenSearch. Die Beziehung zwischen den Anbietern der beiden Services ist mehr als angespannt, und der zu den beiden Produkten zu findende Diskurs ist stark polarisiert, was einen sachlichen Vergleich der beiden Systeme, sowie Ausblicke in deren Zukunft erschwert. Ich habe versucht, Informationen zu der Größe der jeweiligen Nutzer:innencommunity, den zur Verfügung gestellten Features, der vorraussichtlichen Preisstabilität und Verfügbarkeit und Qualität von kompatibler, verwandter Software wie Visualisierungstools zu finden. 
Gegen Ende des Tages hat mein Mitarbeiter mir noch einen Vortrag von den Entwicklern einer dritten Alternative geschickt, welchen ich mir angehört und die wichtigsten Punkte festzuhalten versucht habe. Ich konnte durch diesen Vortrag auch ein besseres Verständnis für die Funktionsweise und Bedienung von Log-Monitoring Systemen entwicklen, insbesondere für die Anforderungen an diese und deren Größe.
