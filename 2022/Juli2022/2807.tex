\subsection{28. Juli}
Heute habe ich versucht, die Beispielanwendung die ich zur Illustration des Metrikexports aus PHP-Anwendungen gebaut hatte in einem Docker-Container zur Verfügung zu stellen. Hierbei bin ich jedoch in einige Probleme gelaufen. Mein Ziel war es, die Anwendung mithilfe von zwei Containern die NGINX und PHP-FPM enthalten zu deployen. Hierbei habe ich den Originalcode als Volume verwendet. Dies hat jedoch zu einigen Problemen geführt. Mein Mitarbeiter musste mir bei der Behebung dieser helfen. Es stellte sich heraus, dass das Problem durch zwei Fehler ausgelöst wurde: Erstens hatten andere Nutzer:innen (was den docker daemon mit einschließt) keine Berechtigungen, den Ordner der meine Codebasis enthält auszuführen (was im Kontext von Linux auch ein "cd" beinhaltet). Hierdurch wurden einige nur schwer nachvollziehbare Fehlermeldungen erzeugt. Weiter habe ich etwas zu naiv einige Konfigurationssnippets für nginx kopiert - unter Anderem beinhalteten diese das Parsing der URL mithilfe von RegEx, welches für den Fall einer auf ".php" endenden URL abgestimmt war. Hierdurch wurden einige nicht existierende Dateinamen zu öffnen versucht. Alles in allem hat sich das Deployment meiner Anwendung dadurch stark verzögert. Ich habe außerdem Dashboards in Grafana erstellt, um die Vorteile der Datenvisualisierung und des Monitoring applikationsinterner Metriken zu showcasen.