\subsection{03. Juni}
Heute habe ich begonnen, eine Unteraufgabe zu einer größeren User-Story zu bearbeiten. In dieser Unteraufgabe ging es darum, im TYPO3 backend eine serviceklasse anzuknüpfen, die bei Druck auf einen button einen slug (also ein verkürzte URL) generiert. Die Anknüpfung gescha konfigurativ über TCA. Anschließend habe ich mich mit dem Bau der relevanten KLasse befasst. Hierzu habe ich das übergebene Requestobjekt analysiert und versucht Logik zu implementieren, die den Slug gemäß der Vorgaben zusammensetzt. Hierbei bin ich durch die Forderung darauf, dass teile des Slugs lokalisiert werden müssen, die Lokalisierungen hierfür jedoch anstatt in xlf Dateien als zusätzliche Datensätze in der Datenbank abgelegt sind in Probleme gelaufen.
Ich konnte einen Großteil der Probleme lösen, musste dafür allerdings ein Repository verändern, was sich wie eine zu tiefgreifende Änderung für solch eine simple Aufgabe anfühlt.