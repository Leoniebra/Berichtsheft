\subsection{17. Juni}
Heute ist unsere Gruppe bei der Bearbeitung unseres Projektes auf das Problem gestoßen, dass wir keine out-of-the-box Lösung zur Messung der von der Solarzelle abgegebenen Stromstärke haben. Im Gegensatz zu Messgrößen wie Luftdruck, Feuchtigkeit, Lautstärkepegel und Lichtintensität haben wir in den zur Verfügung gestellten Baukästen keine Sensoren zur Spannungsmessung. Wir haben Alternativen mit Lehrkräften und Mitschüler:innen besprochen. Dabei haben wir gelernt, dass wir einen Arduino Mikrokontroller verwenden können, um eine Spannung zu messen, woraus wir mit Kenntnis der verwendeten Schaltung die Stromstärke ermitteln können werden. 