\subsection{21. September}
Heute habe zunächst mit der Korrektur eines von mir gestellten Pull-Requests begonnen. Dabei habe ich erneut gemerkt, dass ich noch sehr viel genauer auf die Einhaltung von Coding-Conventions und Dinge wie die korrekte Formulierung von Fehlermeldungen und Benennung von Methoden achten muss. Auch die Strukturierung meines Codes und Anwendung des DRY-Prinzips sind Dinge, über die ich noch einiges zu lernen habe. Hierfür ist Feedback in Form von Kommentaren in einer Pull-Request sehr hilfreich, weil es sich direkt an meinen Versuchen orientiert, und darauf abzielt die von mir gelegte Grundlage verwertbar zu machen. Nachdem ich die Bearbeitung dieser Pull-Request abgeschlossen habe, habe ich mich weiter mit dem Bug-Fix, an dem ich gestern hängen geblieben war beschäftigt. Hierbei habe ich im Gespräch mit zwei Kollegen gelernt, dass über Entity Framework abgeschickte queries an die Datenbank auf mit Varchar (bzw String) besetzten Feldern nur Groß-Kleinschreibungs insensitive Abrufe durchführen. Dies hatte in diesem speziellen Fall zu einem Problem in einem späteren Abgleich geführt. Weiter hatte der gleiche Prozess ein weiteres Problem, das durch die zeitgleiche Ausführung mehrerer asynchroner Operationen auf dem gleichen Datenbankkontext ausgelöst wurde. Ich habe dieses Problem durch Serialisierung der Datenbankaufrufe beheben können. Im Anschluss an diese Arbeit habe ich im Team-Meeting meine versuchte Dokumentation meines Lernprozesses in Confluence vorgestellt. Schließlich habe ich an einer Einführung in die Finanzabteilung von Plan International, und direkt im Anschluss an meiner Einschulung bei der ITECH teilgenommen.