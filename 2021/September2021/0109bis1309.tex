\subsection{1. September - 13. September}
\textit{In diesem Zeitraum habe ich noch keine täglichen Berichte erstellt. Ich habe am Dienstag, dem 14.09. mit der täglichen Dokumentierung meiner Aktivitäten begonnen, und zu diesem Zeitpunkt eine zusammenfassung meiner Tätigkeiten für diesen Zeitraum erstellt.}

 Meine Ausbildung wurde im Anschluss an zwei-wöchiges Praktikum begonnen. Während des Praktikums habe ich ein Einführungsprojekt erstellt, das Grundlagen von SQL, C$\#$, Rest API, Java Script, HTML und CSS in einer einfachen Anwendung, die Datenbankabfragen durch eine Website sendet implementiert hat. Konkreter habe ich eine Datenbank in SQL angelegt, und diese Datenbank mithilfe von Entity Framework an eine Rest-API gekoppelt. Anschließend habe ich eine HTML-Website erstellt, die mithilfe von Fetch-API die Rest-API durch HTTP ansprechen konnte um von mir definierte Operationen auf der Datenbank auszuführen.

Zu Beginn der Ausbildung fand eine Einarbeitung in die Arbeitsstrukturen der Abteilung statt. Eine Einarbeitung in die verschiedenen Unterabteilungen von Plan International, sowie eine branchenspezifische Sicherheitsbelehrung wird die ersten zwei Monate der Ausbildung in Form verschiedener Seminare, Online-Lehrgänge und Einführungsrunden begleiten.

Das erste Projekt in der tatsächlichen Ausbildung beinhaltete die Auseinandersetzung mit dem Arbeitscode der Abteilung, das Verständnis des verwendeten Workflows und der Organisations- und Arbeitssoftware der Abteilung. In dieser Zeit habe ich mich mit objektorientierter Programmierung, insbesondere mit der Nutzung von Entity Framework in der .Net Entwicklung und deren direkten Anwendung im Arbeitscode der Abteilung auseinandergesetzt. Hier habe ich bemerkt, dass mein kurzes Einführungsprojekt während meines Praktikums mir schon einige grundlegende Konzepte gut beigebracht hatte. Der tatsächliche Arbeitscode ist jedoch um mehrere Größenordnungen komplexer.

Ich habe gelernt, Unit-Tests zu schreiben und zur laufenden Qualitätssicherung des Codes zu benutzen. Hierbei hat das Lernen verschiedener Mock-Funktionen und der Philosophie des Unit-Testens einen großen Teil meiner Zeit in Anspruch genommen. Ich habe mich durch verschiedene Ansätze langsam einer effizienten Teststrategie genähert, und die Schritte die ich in der Zwischenzeit unternehmen musste, um zu mockende Module und Services zu verstehen hat sehr zu meinem Verständnis des Codes beigetragen.

Darüber hinaus habe ich die abstraktere Arbeitsweise der Abteilung durch Teilnahme an verschiedenen täglichen und wöchentlichen Meetings kennen gelernt. Dadurch konnte ich bereits ein besseres Verständnis für die verschiedenen bearbeiteten Aufgaben, sowie die Tätigkeiten meiner Kolleg:innen erlangen. Auch die Zusammenarbeit meiner Abteilungen mit verschiedenen anderen Unterabteilungen ist hierdurch klarer geworden.

Ein großer Teil meiner Tätigkeit in dieser Zeit war neben dem Lernen der verwendeten Code-Konzepte und Softwares auch das Lernen der internen Code-Konventionen zur Wiederverwendbarkeit und Lesbarkeit des geschriebenen Codes.
Ich wurde außerdem in die Nutzung von Git zur Versionskontrolle eingearbeitet, und habe es in diesem Zeitraum auch geschafft, eine Lösung als Pull-Request an den Arbeitsbranch der Abteilung zu stellen, die von meinen Kolleg:innen gestellten Korrekturen zu bearbeiten und erfolgreich einen Beitrag zum aktuellen Projekt zu leisten.
%%%%%%%%%%%%% Index %%%%%%%%%%%%%%%%%

