\subsection{03. November}
Heute habe ich mich weiter mit der Überarbeitung der Diagramme zur Abbildung unserer Systemlandschaft beschäftigt. Mein Plan für heute war, mich von der Fachlichen ebene zu entfernen, und immer technischer werdende Diagramme zu erstellen. Hierbei bin ich jedoch auf das große Problem gestoßen, dass ich den überwiegenden Teil der Systemlandschaft erst durch eben jene grafischen Darstellungen kennen gelernt habe die ich zu überarbeiten versuche. Ich habe dadurch große Schwierigkeiten darin erlebt, diesen Darstellungen Informationen hinzuzufügen, oder zu entscheiden welche Informationen einer klaren Darstellung dienen und welche überflüssig sind. Abgesehen von der Anwendung grundlegender Prinzipien grafischer Darstellungen, wie Lesbarkeit, Einfachheit und visueller Konsistenz habe ich heute nur vernachlässigbare Fortschritte machen können. Ich konnte jedoch an einigen wenigen Stellen durch gezielte Nachfragen bei meinen Kolleg:innen zumindest mein eigenes Verständnis isolierter Prozesse verbessern, und einen tieferen Einblick in die Funktionsweise des Projektes erlangen. Leider sind diese Einblicke zu schmal und konkret gewesen, um für eine abstrakte Grafik von Nutzen zu sein. Zum Ende des Arbeitstages habe ich erneut mit meinem Vorgesetzten die Zielsetzung besprochen und dabei konkretisieren können, sodass ich selbstständig an der Aufgabe weiterarbeiten können werde.
