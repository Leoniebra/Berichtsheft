\subsection{21. Dezember}
Heute habe ich in der Berufsschule einige weitere meiner neuen Lehrer:innen kennengelernt. Ich habe an einem input zu IPv4 teilgenommen. Dabei fand ich die Berechnung von IP-Adressen aus der Netzwerk-IP und der Netzwerkmaske besonders spannend. Die Dekomposition der IP in Basis 2 und darauffolgende und-Verknüpfung der Koeffizienten bildet auf dem Raum der verfügbaren IP-Adressen eine Monoidstruktur, in welcher jedes Element idempotent ist. Weiter ist die 1 (mit dieser Verknüpfung ist die 1 gegeben durch 255.255.255.255) das einzige invertierbare Element, jedes andere Element ist ein Nullteiler. Es ist verlockend, zu versuchen diese Monoidstruktur zu einer Ringstruktur auf den ganzen Zahlen zu erweitern, jedoch muss diese Ringstruktur einen Ring ohne 1 ergeben. Wir können diese Verknüpfung jedoch nutzen, um den Raum der Folgen in $\mathbb{F}_2$ mit einer Ringstruktur zu versehen (unter Verwendung der aus dem Rechnen mit irrationalen reellen Zahlen in Binärdarstellung bekannten Addition):
\begin{align*}
          \mathrm{Set}(\mathbb{N},\mathbb{F}_2)  \times \mathrm{Set}(\mathbb{N},\mathbb{F}_2) \hspace{.5cm}& \longrightarrow  \hspace{1cm}\mathrm{Set}(\mathbb{N},\mathbb{F}_2)  \\
        \big{(} (a_0, a_1, a_2, \hdots), (b_0, b_1, b_2, \hdots) \big{)} & \longmapsto  (a_0\wedge b_0, a_1\wedge b_1, a_2 \wedge b_2, \hdots)
\end{align*}
