\subsection{11. Oktober}
Heute habe ich mich selbstständig mit der Erzeugung und Abfrage der Daten für die Kreuztabellendarstellung eines Cashflow-Planes beschäftigt. Ich bin dabei in einige Probleme gelaufen, weil der Aufruf von Modulen, die Abfragen an die Datenbank senden innerhalb verschachtelter Schleifen zu Komplikationen führte. Diese wurden dadurch ausgelöst, dass die Methoden der Module mit Parametern aufgerufen wurden, die innerhalb der Schleifen geändert wurden. Dazu waren die Parameter zur Laufzeit nicht sicher zur Verfügung gestellt, was zu unerwartetem Verhalten der Datenbankabfragen führte. Das Problem konnte dadurch behoben werden, dass die Parameter als Werte in neue Variablen kopiert wurden, und diese dummy-Variablen als Parameter übergeben wurden. Ein weiteres Problem wurde dadurch ausgelöst, dass die Daten auf meiner Testumgebung veraltet und inkonsistent waren. Dadurch existierten einige Referenzen auf gelöschte Datentypen, wodurch eine Flut von Null-Reference-Exceptions ausgelöst wurde. Dieses Problem habe ich durch eine Anpassung der Testdaten zur besseren Modellierung des Live-Systems beheben können.