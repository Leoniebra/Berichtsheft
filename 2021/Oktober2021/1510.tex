\subsection{15. Oktober}
Heute habe ich mit einem Mitarbeiter seine Kommentare zu der von mir gestellten Pull-Request besprochen. Dabei habe ich bemerkt, dass ich immer noch zu wenig Wert auf die Lesbarkeit und Nachvollziehbarkeit des von mir geschriebenen Codes lege, und mir zur Verfügung stehende Tools zur Sicherstellung der Code-Qualität nicht ausreichend nutze. Wir haben die von ReSharper bereitgestellte Beurteilung des Codes, sowie die für unsere Abteilung hinterlegten Coding-Guidelines besprochen. Weiter haben wir die Art und Weise, auf die ich Methoden baue diskutiert. Ich habe noch die schlechte Angewohnheit, verschachtelte Schleifen über harte Indizierungen mehrdimensionaler Objekte zu schreiben. Das entspricht nicht den modernen Standards, weil diese Praxis fehleranfällig, sowie schwer lesbar und nachvollziehbar ist. Tatsächlich ließen sich viele Schleifen die ich verwendet hatte als einmalige LINQ-Select aufrufe refaktorisieren. 

Im Rahmen der Diskussion meines Pull-Requests habe ich mehrere Artikel über Coding-Conventions auf Stackoverflow gelesen. Dadurch habe ich einiges über Coding-Conventions im Allgemeinen und die Verwendung von Regions in C\# im Speziellen gelernt, und die Wichtigkeit von Refaktorisierung und Modularisierung des geschriebenen Codes eingesehen. Ich finde dieses Thema sehr interessant, und hoffe mich darin im Laufe meiner Ausbildung noch signifikant zu verbessern. Ich habe in der vergangenen Woche gemerkt, dass selbst ich beim Lesen meines eigenen Codes Probleme hatte meiner eigenen Logik zu folgen.

Nachdem ich meinen Code aufgeräumt und vereinfacht hatte (die PR wird trotzdem nicht durchgehen, weil es einige Funktionalitäten gab die ich nicht implementieren konnte), habe ich selbst eine Pull-Request meines Mitarbeiters unterseucht, und versucht dort Kommentare zu hinterlassen.