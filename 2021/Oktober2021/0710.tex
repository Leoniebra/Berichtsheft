\subsection{07. Oktober}
Heute habe ich einem Kollegen meinen bisherigen Fortschritt gezeigt, und er hat mir einige Tipps zur weiteren Strategie gegeben. Auf seinen Rat hin habe ich mich weg von der Datenstruktur und hin zur konkreten Einbindung orientiert, um mir selbst einen Rahmen zu schaffen in dem ich mit der Debug-Funktion arbeiten kann. Dies habe ich im Laufe des Tages auch erfolgreich umsetzen können, und kann nun die geforderte Kreuztabelle grafisch darstellen und mit der Debug-funktion in meine Methoden zur Datengenerierung einsteigen. Ich habe außerdem am Sprint-Review als Zuhörerin teilgenommen, und die Präsentation der Resultate des vergangenen Sprints gegenüber den Stakeholder:innen mitangesehen. Dies war sehr spannend für mich, weil es den Zweck vieler unserer Tätigkeiten in einen besseren Kontext gesetzt hat. Schließlich habe ich auch am Sprint-Planning Meeting teilgenommen, und hierbei versucht mich aktiver einzubringen und ein besseres Verständnis für die Komplexitätsschätzung von Stories im Scrum-Prozess zu erlagen.